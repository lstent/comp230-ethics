\documentclass{scrartcl}

\usepackage[hidelinks]{hyperref}
\usepackage[none]{hyphenat}

\title{Digital Game Piracy: Criminals or Archivers?}

\subtitle{COMP 230 - Ethics \& Professionalism Essay}

\author{1506919}

\begin{document}

\maketitle

\abstract{This paper discusses digital game piracy of abandonware/ orphan works. First stating what the law on digital game piracy is and looking at the research on how damaging it is to game studios, how much of the worlds population takes part in it, some of the possible reasons why individuals pirate digital games.

The topic of pirating abandonware is then explored, giving a brief definition of both orphan works and abandonware, why digital game companies abandon titles, legal methods of archiving and the effect piracy has on archiving digital games.

The conclusion summarises all the information, weighing the benefits against the drawback of piracy to find if individuals deserve to be labelled as criminals.}

\section*{Introduction}

Digital game piracy is condemned by law as a very serious crime, ``which is now widespread, decentralised problem in which millions of people take part'' \cite{myles2006content}. Yet reliable data recording the amount of individuals taking part in pirating, which games they downloaded and information such as if that game is available in that country or is that game abandonware are rare.

G. Myles and S. Nusser wrote the paper `Content Protection for Games' \cite{myles2006content} in which they state ``The industry lost more than \$1.8 billion to global piracy'' (2004), the reference for this statement is the title ``2005 special 301 report of global copyright protection and enforcement'' and the iipa website, which lists a number of links \cite{iipa}, one is a site that calculates global piracy by counting the amount of traffic on sites with the option to stream, download or torrent digital contents \cite{muso}.

A google search for the title results in the PDF `2005 special 301 report executive summary' \cite{301report} that does not mention 1.8 million and does not provide reference or evidence to any estimated losses.

Both the report and website seem to unreliable to base statements on the amount of money lost to global piracy.

\section*{The Law}

The Oxford dictionary describes software piracy as unauthorised use or reproduction of another's work.\cite{dictionary2007oxford}

Though UNESCOs definition of internet piracy seems more appropriate ``peer - to - peer (P2P) file sharing termed ``piracy'', even if an economical motive may not be present.'', the main offence is the damage to the company profit.

The common punishment for internet piracy is having to reimburse the profits lost to the company/ rights holder including the cost of legal actions, but depending on which country the offence took place the crime may end in a prison sentence ranging from 6 months to 10 years.\cite{panethiere2005persistence}

\section*{Damages and possible root causes of digital game piracy}

The most obvious damage from piracy is the loss of profit by the game company, though there seems to be no full proof way to calculate loss, the paper Balancing Video Game Piracy\cite{gorder2004balancing} looked into the amount of players using a pirated copy of the game Wizardry\cite{sir-tech1981wizardry}, they found depending on which consulting firm asked ``anywhere from 30 to 50 percent were using a pirated copy'' even though ``Wizardry was an early financial success''.

The paper Distribution of digital games via BitTorrent\cite{drachen2011distribution} monitored P2P downloads of 173 titles over 3 months and found that the most common games downloaded had a higher age rating, which could mean that this is a way players below that age get access to games they can't legally obtain, they also found games with a high meta-rating score were pirated more, this could point to price being a reason for pirating, this study concentrated on well known titles produced by AAA companies that also come with a large price tag, so those individuals that can't afford the game can still get a copy.

\section*{Abandonware/ Orphan Works}

Computer software including digital games that are no longer produced, distributed and supported by the copyright holder is called abandonware.

The term orphan works refers to software who's copyright owners are unknown or cannot be found therefore legal copies cannot be obtained.\cite{khong2006orphan}


There are three main types of abandonment based on why the copyright holder abandoned the software.

Stratigic abandonment happens when the copyright holder sells an upgraded version of the same product, to relate this to the digital game industry every new update to a game is a new version and the previous unchanged version becomes abandonware.

Temporary abandonment occurs when the copyright holder stops production of their digital game to continue production at a later date, usually to raise the value.

Commercial abandonment is the most obvious reason, when a digital game costs more to produce than it returns in sales. 

According to the U.S. Copyright Office ``copyright protection lasts 95 years after its first publication or 120 years after creation, whichever expires first'' \cite{gov}.

Therefore if a game becomes abandonware there is no legal way to obtain a copy until the copyright dissolves, at which point no copies could exist or systems have advanced and the game can no longer be played, with no legal archeiving system in place these games are lost\cite{gooding2008grand}.

The largest source of abandonware games is through the pirating community ``including the full emulation of old games for new systems''\cite{gooding2008grand}.

The article Games in Captivity discusses that in the case of abandonware ``copyright law is no longer serving its purpose and has been warped to serve only the interests of multi-national coporations instead of authors or the public''\cite{barton2005games}. Though it is important to note corporations make no profit from abandonware either just that it doesn't require any effort or cost.

Searching abandonware online results in many sites that abandonware games can be downloaded for free of which the claim do notharm the gaming industry `` There is no alternative to downloading them. You cannot buy them anywhere''\cite{barwick2008barriers}.

\section*{Conclusion}

Pirating commercial abandonware is different to any other form of piracy as there is no other way to access these games, instead of the abandoned games being forgotten a solution should be put in place that either the copyright is also abandoned when the game is or when a game is abandoned that gives permission for the game to be free to play for everyone and to be archieved by both official and unofficial websites with a link to download.

\bibliographystyle{ieeetran}
\bibliography{References}

\end{document}
